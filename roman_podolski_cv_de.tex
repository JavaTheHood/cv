% arara: xelatex
%%%%%%%%%%%%%%%%%%%%%%%%%%%%%%%%%%%%%%%%%
% Plasmati Graduate CV
% LaTeX Template
% Version 1.0 (24/3/13)
%
% This template has been downloaded from:
% http://www.LaTeXTemplates.com
%
% Original author:
% Alessandro Plasmati (alessandro.plasmati@gmail.com)
%
% License:
% CC BY-NC-SA 3.0 (http://creativecommons.org/licenses/by-nc-sa/3.0/)
%
% Important note:
% This template needs to be compiled with XeLaTeX.
% The main document font is called Fontin and can be downloaded for free
% from here: http://www.exljbris.com/fontin.html
%
%%%%%%%%%%%%%%%%%%%%%%%%%%%%%%%%%%%%%%%%%
 
%----------------------------------------------------------------------------------------
%	PACKAGES AND OTHER DOCUMENT CONFIGURATIONS
%----------------------------------------------------------------------------------------
 
\documentclass[a4paper,10pt]{article} % Default font size and paper size
 
\usepackage{fontspec} % For loading fonts
\defaultfontfeatures{Mapping=tex-text}
\setmainfont[SmallCapsFont = Lato]{Lato} % Main document font
 
\usepackage{xunicode,xltxtra,url,parskip} % Formatting packages
 
\usepackage[usenames,dvipsnames]{xcolor} % Required for specifying custom colors
\usepackage{fontawesome} % To use all the cool logos
 
\usepackage[big]{layaureo} % Margin formatting of the A4 page, an alternative to layaureo can be \usepackage{fullpage}
% To reduce the height of the top margin uncomment: \addtolength{\voffset}{-1.3cm}
 
\usepackage{hyperref} % Required for adding links	and customizing them
\definecolor{linkcolour}{rgb}{0,0,0} % Link color
\hypersetup{colorlinks,breaklinks,urlcolor=linkcolour,linkcolor=linkcolour} % Set link colors throughout the document
\usepackage{titlesec} % Used to customize the \section command
\titleformat{\section}{\Large\scshape\raggedright}{}{0em}{}[\titlerule] % Text formatting of sections
\titlespacing{\section}{0pt}{3pt}{3pt} % Spacing around sections
 
\begin{document}
 
\pagestyle{empty} % Removes page numbering
 
%----------------------------------------------------------------------------------------
%	NAME AND CONTACT INFORMATION
%----------------------------------------------------------------------------------------
 
\par{\centering{\Huge Roman C. \textsc{Podolski}}\bigskip\par} % Your name
 
\section{Persönliche Daten}
 
\begin{tabular}{l@{: }l}
%\textsc{Geburtsort | Geburtsdatum} & München, Deutschland  | 07 Dezember 1989 \\
\textsc{Geburtsdatum} & 07. Dezember 1989 \\
\textsc{Addresse}     & Kellerstraße 8a, 81667 München, Deutschland \\
\textsc{Telefon} \faPhone      & (+49) 178 81 33 292\\
\textsc{E-Mail}  \faEnvelope     & \href{mailto:roman.podolski@tum.de}{roman.podolski@tum.de}
\end{tabular}
 
%----------------------------------------------------------------------------------------
%	EDUCATION
%----------------------------------------------------------------------------------------
 
\section{Studium und Schulbildung}
 
\begin{tabular}{r|l@{: }p{11cm}}
%\begin{tabular}{rp{11cm}}
\emph{Aktuell} & \multicolumn{2}{p{11cm}}{Master of Science in \textsc{Robotic, Cognition, Intelligence}}\\
\textsc{04/2015 - 04/2017} & \multicolumn{2}{p{11cm}}{\textbf{Technische Universität München}}\\
\multicolumn{3}{c}{}\\
& \multicolumn{2}{p{11cm}}{Bachelor of Engeneering in \textsc{Geotelematik \& Navigation}}\\
\textsc{10/2010 - 04/2015} & \multicolumn{2}{p{11cm}}{\textbf{Hochschule München}, München} \\
                           & \small Bachelorgesamtnote & \small  2.5\\
                           & \small Bachelorarbeit & \small ``Serverseitiges Post Processing von GNSS Rohdaten'' \\
                           & \small Note           & \small 2.0\\
			                     & \small Betreuer       & \small \href{mailto:lars.wischhof@hm.edu}{Prof. Dr.-Ing. Lars Wischhof (lars.wischhof@hm.edu}) \\
                           & \small Externe Stelle & \small \href{http://www.audi-electronics-venture.de}{Audi Electronics Venture GmbH} \\
                           & \multicolumn{2}{p{11cm}}{\small Implementation serverseitiger Bayes-Filteralgorithmen zur Sensordatenfusion, von GNSS und IMU Rohdaten als Funktionsbibliothek für das Realtionale Datenbanksystem PostgreSQL.}\\
\multicolumn{3}{c}{}\\
 
%------------------------------------------------
 
\textsc{08/2007 - 08/2009} & \multicolumn{2}{p{11cm}}{\textbf{Kath. Romano-Guardini-Fachoberschule für Sozialwesen}, München} \\
                           & \multicolumn{2}{p{11cm}}{\small Fachhochschulreife: 2.7}                                         \\
\multicolumn{3}{c}{}                                                                                \\
 
%------------------------------------------------
 
\textsc{08/2001 - 08/2007} & \multicolumn{2}{p{11cm}}{\textbf{Städt. Fridjof-Nansen-Realschule}, München} \\
                           & \multicolumn{2}{p{11cm}}{\small Realschulabschluss: 2.0} \\
\multicolumn{3}{c}{}\\
 
%------------------------------------------------
 
\textsc{08/1999 - 08/2001} & \multicolumn{2}{p{11cm}}{\textbf{Städt. Maria-Therisia-Gymnasium}, München} \\
\multicolumn{3}{c}{}\\
\end{tabular}
 
%----------------------------------------------------------------------------------------
%	COMPUTER SKILLS 
%----------------------------------------------------------------------------------------
 
%\section{Fähigkeiten und Kenntnisse}
 
%\begin{tabular}{ll}
%\textbf{Programmiersprachen und Datenbanken}   & \textbf{Algorithmen und Verfahren}\\
%C/C++                                 & Kalman Filter \\
%Java                                  & Koordinaten Transformationen \\ 
%MATLAB                                & Kamera Kalibrierung und Modellierung \\ 
%Ruby                                  & Integral Transformationen  \\ 
%Javascript / Coffeescript             & \\ 
%PostgreSQL                            & \\ 
                                      %& \\ 
%\textbf{Sensoren zur Anwendungsprogrammierung} & \textbf{Weitere Kenntisse}\\
%GPS                                   & LaTeX, XeLaTeX \& Ti\textit{k}Z\\ 
%IMU                                   & Git \\ 
%Laserscanner                          & Subversion \\ 
%UWB                                   & SCRUM \\ 
                                      %& Vim \\
                                      %& TDD \ BDD \ FDD
%\end{tabular}
 
%----------------------------------------------------------------------------------------
%	WORK EXPERIENCE 
%----------------------------------------------------------------------------------------
 
\section{Tätigkeiten}
 
\begin{tabular}{r|p{11cm}}
%\begin{tabular}{rp{11cm}}
 
\emph{Aktuell}             & Werkstudent, \textbf{ESRLabs AG}, München\\
seit \textsc{10/2015}      & \emph{Softwareentwickler}\\
                           & \small{Design und Entwicklung von Soft- und Hardware basierten Testsystemen für Automotive und Embedded Anwendungen} \\
\multicolumn{2}{c}{} \\

 \textsc{06/2014 - 10/2015}           & Werkstudent, \textbf{Uniscon GmbH}, München\\
      & \emph{Softwareentwickler}\\
                           & \small{Entwicklung einer AngularJS basierten Webanwendung in einem agilen Team.} \\
\multicolumn{2}{c}{} \\
\textsc{08/2012 - 10/2013} & Interdisziplinäres Praxis Semester \textbf{Uniscon GmbH}, München\\
                           & \emph{Softwareentwickler}\\
 & \small{Entwurf und Umsetzung von Strategien zur Automatisierung von Software-Tests unterschiedlicher Client-Applikationen. Management eines Entwickler-Teams in Indien.} \\
\multicolumn{2}{c}{} \\
 
%------------------------------------------------
 
\textsc{06/2013}           & Satelliten-Positionierung \textbf{Hochschule München}, München\\
                           & \emph{Tutor}\\
                           & \small{Betreuung und Evaluierung von Studienarbeiten in MATLAB.}\\
\multicolumn{2}{c}{} \\
 
%------------------------------------------------
 
\textsc{09/2013}           & Softwareentwicklung \textbf{Hochschule München}, München\\
\emph{und}                 & \emph{Tutor}\\
\textsc{09/2012}           & \small{Konzeption, Durchführung und Betreuung eines Vorbereitungskurses für Studienanfänger, Thema: ``Softwareentwicklung in Java''.}\\
\multicolumn{2}{c}{} \\
 
%------------------------------------------------
 
\textsc{09/2009 - 08/2010} & Freiwilliges Ökologisches Jahr \\
                           & \textbf{Jugendorganisation Bund Naturschutz}, München\\
                           & \emph{Bildungsarbeit}\\
                           & \small{Organisation und Betreuung von Projekten zu den Themengebieten Nachhaltigkeit und Ökologie.}\\
\multicolumn{2}{c}{} \\
 
%------------------------------------------------
\end{tabular}
 
%----------------------------------------------------------------------------------------
%	SCHOLARSHIPS AND ADDITIONAL INFO
%----------------------------------------------------------------------------------------
 
%\section{Scholarships and Certificates}
 
%\begin{tabular}{rl}
%\textsc{Sept.} 2012 & Faculty of Science Masters Scholarship \footnotesize(\$30,000)\normalsize\\
 
%\textsc{June} 2010 & {\textsc{Gmat}\textregistered}\setmainfont[SmallCapsFont=Fontin SmallCaps]{Fontin-Regular}: 730 (\textsc{q:50;v:39}) 96\textsuperscript{th} percentile; \textsc{awa}: 6.0/6.0 (89\textsuperscript{th} percentile)
%\end{tabular}
 
%----------------------------------------------------------------------------------------
%	LANGUAGES
%----------------------------------------------------------------------------------------
 
\section{Sprachen}
 
\begin{tabular}{rl}
\textsc{Englisch:} & Verhandlungssicher\\
 
\textsc{Deutsch:} & Muttersprache\\
 
  \textsc{Spanisch:} & Grundkenntnisse (A2)\\
\end{tabular}
 
 
%----------------------------------------------------------------------------------------
%	INTERESTS AND ACTIVITIES
%----------------------------------------------------------------------------------------
 
\section{Interessen}
 
\href{https://github.com/RomanCPodolski}{Open-Source (\faGithub https://github.com/RomanCPodolski/)}, Robotik, Künstliche Intelligenz, Maschinelles Lernen,
Boxen, Wandern, Fitness, Video- und Brettspiele, Schach
 
%----------------------------------------------------------------------------------------
 
\newpage
 
\end{document}
